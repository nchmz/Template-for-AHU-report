% TEX program=xelatex
\documentclass[experiment]{ahu_report}

\setbgopacity{0.4}

\stuid{P12314XXX}
\major{专业名称}
\name{电子出森}
\expdate{X月X日}
\course{课程名称}

\begin{document}
\begin{spacing}{1.25}
    \makeahutitle
  \begin{center}
    {\zihao{2}\songti 实验名称}
  \end{center}
\section{实验目的}
\begin{enumerate}
    \item 掌握本实验相关的基础理论知识。
    \item 熟悉实验仪器的使用方法。
    \item 能够独立完成实验操作并记录数据。
\end{enumerate}

\section{实验原理}
这里是实验原理部分。请在这里阐述实验所依据的理论基础、计算公式或工作原理。

\subsection{原理小节示例}
可以分小节进行详细描述。例如,介绍实验中涉及的硬件电路或算法流程。

\section{图片插入示例}
本模板支持插入单张图片和多张子图。建议将图片文件放在 \texttt{pic} 文件夹中。

\subsection{单张图片}
\begin{figure}[H]
    \centering
    \includegraphics[width=0.6\textwidth]{pic/ex1.png}
    \caption{单张示例图片}
    \label{fig:ex1}
\end{figure}

\subsection{多张子图(Subfigure)}
以下是并排显示多张子图的示例,使用了 \texttt{subfigure} 宏包。

\begin{figure}[H]
\centering
\subfigure[电解电容]{
\includegraphics[width=0.3\linewidth]{pic/cap1.png}
}
\hfill
\subfigure[陶瓷电容]{
\includegraphics[width=0.3\linewidth]{pic/cap2.png}
}
\hfill
\subfigure[钽电容]{
\includegraphics[width=0.3\linewidth]{pic/cap3.png}
}
\caption{实验中使用的电容实物图(子图示例)}
\label{fig:caps}
\end{figure}

\begin{figure}[H]
\centering
\subfigure[三极管驱动电路示意]{
\includegraphics[width=0.45\linewidth]{pic/transistor_usage.png}
}
\hfill
\subfigure[PNP与NPN三极管]{
\includegraphics[width=0.45\linewidth]{pic/transistor_types.png}
}
\caption{三极管驱动电路与实物图(两张子图示例)}
\label{fig:transistors}
\end{figure}

\section{代码块插入示例}
本模板支持插入代码块,并支持语法高亮。以下是 C 语言代码的示例:

\begin{lstlisting}[language=C, caption={C语言代码示例}, label={code:c_example}]
#include <stdio.h>

int main() {
    printf("Hello, AHU!\n");
    return 0;
}
\end{lstlisting}

也可以插入 Python 代码:

\begin{lstlisting}[language=Python, caption={Python代码示例}]
def hello():
    print("Hello, AHU!")

if __name__ == "__main__":
    hello()
\end{lstlisting}

\section{实验内容与步骤}
\begin{enumerate}
    \item 搭建实验环境,连接实验线路。
    \item 编写或运行实验代码/程序。
    \item 观察实验现象,记录实验数据。
\end{enumerate}

\section{实验结果与分析}
这里展示实验结果,包括数据表格、波形图或截图,并对结果进行分析讨论。

\end{spacing}
\end{document}
