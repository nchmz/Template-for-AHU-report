%  TEX program=xelatex
\documentclass[a4paper,UTF8]{article}
% \setlength\paperheight{26cm}
% \setlength\paperwidth{18.4cm}
\usepackage[fontset=windows,zihao={-4}]{ctex} % Chinese support, using Windows fonts
\usepackage{setspace} % 导入 setspace 宏包
\usepackage{graphicx} % Insert images
\usepackage{listings} % Print source code
\usepackage{color} % Color support
\usepackage{amsmath} % 数学公式支持
\usepackage{booktabs} % Professional table support
\usepackage{pdflscape} % Landscape pages support in PDF
\usepackage{hyperref} % Hypertext links support for cross-referencing
\usepackage{geometry} % Page layout customization
\usepackage{float}
\usepackage{subfigure}
\usepackage{wrapfig} % 提供文字环绕图片功能
\usepackage{caption} % 提供 \captionof 命令,用于在非浮动环境中添加标题
\usepackage{fontspec} % 导入 fontspec 宏包,用于设置字体
\usepackage{tikz} % 用于高级图形处理和透明度控制
\usepackage{fancyhdr} % 用于页眉页脚和背景控制
% 
% \geometry{left = 8 cm, right=8cm , top = 3cm, bottom = 3cm}
% Customize hyperref format (it's set to no special format here)
\hypersetup{hidelinks}
% 设置全局页面边距
\geometry{left=3.17cm, right=3.17cm, top=2.54cm, bottom=2.54cm}
% 设置全局英文字体
\setmainfont{Times New Roman} % 你可以将 Times New Roman 替换为你想要的字体
% Declare directories to search for graphics files for graphicx
\graphicspath{{logo/}}
\graphicspath{{pic/}}
% 声明华文宋体以用于标题(依赖 xeCJK/ctex 提供的 \newCJKfontfamily)
\newCJKfontfamily\HWST{华文宋体}

% 设置页面背景(透明度0.1,首页除外)
\newcounter{bgpage}
\setcounter{bgpage}{0}

% 使用 fancyhdr 在每一页添加背景
\pagestyle{fancy}
\fancyhf{}

% 在后台绘制背景 - 必须在使用前定义
\newcommand{\AddBg}{%
  \stepcounter{bgpage}%
  \ifnum\value{bgpage}>0\relax
    \begin{tikzpicture}[remember picture, overlay]
      \node[inner sep=0pt, outer sep=0pt, opacity=0.1, anchor=center] at (current page.center) {%
        \includegraphics[width=\paperwidth, height=\paperheight]{background}%
      };
    \end{tikzpicture}%
  \fi
}

% 页眉居中显示学院与报告名称(替换原来的 leftmark)
\chead{\zihao{-5}\songti 安徽大学电子信息工程学院\quad 实验报告}
% 页码在页脚居中显示(保留背景插入)
\cfoot{\AddBg\thepage}
% 设置页眉线宽度
\renewcommand{\headrulewidth}{0.4pt}


% Define new command for title page
\newcommand{\reporttitle}[2]{
  \LARGE\textsf{#1}\quad\underline{\makebox[12em]{#2}}
}
\newcommand{\reportinfo}[2]{
  \large\makebox[4em]{\textsf{#1}}\quad\underline{\makebox[15em]{#2}}
}

% The document begins here
\begin{document}
\begin{spacing}{1.25} % 设置全局行距为1.25倍

  \newpage
  % 在摘要前插入标题:学院名,下一行为一行带下划线的课程名(微机原理)与同一行的“实验报告”对齐
  \begin{center}
    {\HWST\zihao{-1}\bfseries 安徽大学电子信息工程学院}\\[8pt]
    % 使用 tabular 实现在同一行:左侧为带下划线的课程名,右侧为 实验报告
    \makebox[\linewidth][c]{%
      % 直接对文字使用 \underline,使下划线长度与文字本身一致
      \underline{\HWST\zihao{-1}\bfseries 微机原理}% 左侧下划线文本(长度与文字一致)
      % \hspace{2cm}% 间距
      {\HWST\zihao{-1}\bfseries 实验报告}% 右侧文本
    }\\[8pt]
  \end{center}

  \songti
  % 学生信息栏:三列两行(学号/专业/姓名;实验日期/实验成绩/教师签字)
  \begin{center}
    \vspace{6pt}
    \begin{tabular}{@{}c@{}c@{}c@{}}
      % 第一行:学号、专业、姓名(使用 makebox 居中并去掉标签与线之间的空隙)
  \makebox[0.33\textwidth][c]{\zihao{5}\songti 学号\underline{\HWST\zihao{5} P12314000}} &
  \makebox[0.33\textwidth][c]{\zihao{5}\songti 专业\underline{\HWST\zihao{5} 电子信息工程}} &
  \makebox[0.33\textwidth][c]{\zihao{5}\songti 姓名\underline{\HWST\zihao{5} 电子出森}} \\\vspace{6pt}
      % 第二行:实验日期、实验成绩、教师签字(同上,标签与下划线紧贴)
  % 第一项已填:学号 写在下划线上(P12314101),下划线长度与文字一致
  \makebox[0.33\textwidth][c]{\zihao{5}\songti 实验日期\underline{\HWST\zihao{5} 10月29日}} &
    \makebox[0.33\textwidth][c]{\zihao{5}\songti 实验成绩\underline{\HWST\zihao{5}  }} &
    \makebox[0.33\textwidth][c]{\zihao{5}\songti 教师签字\underline{\HWST\zihao{5}  }} \\
    \end{tabular}
  \end{center}
% \begin{abstract}
% 随着“双碳”目标的提出,核能在我国能源结构优化与低碳转型中扮演着越来越重要的角色。本文基于《中国核能发展报告(2024)》\cite{CNEA2024}及国家能源局相关文件\cite{NEA2023, NEAPlan2007},探讨了我国核电中长期发展规划中的主要限制因素,并从政策、技术与社会角度提出相应对策建议。
% \end{abstract}
  % 实验名称(居中显示)
  \begin{center}
    {\zihao{2}\songti 算术逻辑运算器实验}
  \end{center}

\section{实验目的}
\begin{enumerate}
    \item 深入了解算术逻辑运算器(74LS181)的内部结构、功能特性及其在数字系统中的应用。通过实际操作,掌握该芯片的基本工作原理和功能实现机制。
    
    \item 系统学习基本算术运算(如加法、减法)和逻辑运算(如与、或、非、异或等)在硬件层面的实现方法。理解计算机系统中运算器的核心作用,为后续学习计算机组成原理和体系结构打下坚实基础。
    
    \item 通过对比有符号数和无符号数的运算结果,深入理解补码表示法在计算机运算中的重要性,掌握标志位(如进位标志CF、符号标志SF等)的含义和判断方法。
\end{enumerate}

\section{实验原理}

本实验原理分为两大部分:第一部分介绍实验所需的核心器件及其功能特性;第二部分说明实验电路的组成与工作原理。

\subsection{实验器件}

本实验使用的主要器件如表~\ref{tab:devices}所示。

\begin{table}[H]
\centering
\caption{实验器件清单}
\label{tab:devices}
\begin{tabular}{cll}
\toprule
序号 & 器件名称 & 型号 \\
\midrule
1 & 算术逻辑运算器 & 74LS181 \\
2 & 三态门 & 74LS244 \\
3 & 地址/数据锁存器 & 74LS273 \\
4 & 反相器 & 74LS04 \\
5 & 或/或非门 & 74HC4078 \\
\bottomrule
\end{tabular}
\end{table}

\textbf{1. 算术逻辑运算器(74LS181)}

74LS181是一款经典的4位算术逻辑运算器(ALU),是理解计算机运算器设计的典型器件。


\begin{wrapfigure}{r}{0.35\textwidth}
\centering
\vspace{-10pt}
\includegraphics[width=0.33\textwidth]{74LS181}
\caption{74LS181引脚图}
\label{fig:74LS181}
\vspace{-10pt}
\end{wrapfigure}
其主要功能包括:
\begin{itemize}
    \item 具有两个4位数据输入端(A3-A0, B3-B0)和一个4位数据输出端(F3-F0)
    \item 通过4个功能选择端(S3-S0)和方式控制端(M)选择运算类型
    \item 当M=1时执行逻辑运算(与、或、非、异或等16种逻辑操作)
    \item 当M=0时执行算术运算(加法、减法等16种算术操作)
    \item 提供进位输入(Cn)和进位输出(Cn+4),支持多片级联实现更高位数运算
\end{itemize}

其完整的逻辑功能表如表~\ref{tab:74ls181}所示。
\begin{table}[htbp]
\centering
\caption{74LS181逻辑功能表}
\label{tab:74ls181}
\zihao{-5}
\begin{tabular}{|c|c|c|c|c|c|c|}
\hline
\multicolumn{4}{|c|}{选择输入} & \multicolumn{2}{c|}{M=0(算术运算)} & M=1 \\
\cline{0-6}
S3 & S2 & S1 & S0 & CN=1 无进位 & CN=0 有进位 & (逻辑运算)\\
\hline
0 & 0 & 0 & 0 & $F=A$ & $F=A$ 加 1 & $F=\overline{A}$ \\
\hline
0 & 0 & 0 & 1 & $F=A+B$ & $F=(A+B)$ 加 1 & $F=\overline{A}+\overline{B}$ \\
\hline
0 & 0 & 1 & 0 & $F=A+\overline{B}$ & $F=(A+\overline{B})$ 加 1 & $F=\overline{A}B$ \\
\hline
0 & 0 & 1 & 1 & $F=0$ 减 1 & $F=0$ & $F=0$ \\
\hline
0 & 1 & 0 & 0 & $F=A$ 加 $A\overline{B}$ & $F=A$ 加 $A\overline{B}$ 加 1 & $F=\overline{AB}$ \\
\hline
0 & 1 & 0 & 1 & $F=(A+B)$ 加 $A\overline{B}$ & $F=(A+B)$ 加 $A\overline{B}$ 加 1 & $F=\overline{B}$ \\
\hline
0 & 1 & 1 & 0 & $F=A$ 减 $B$ 减 1 & $F=A$ 减 $B$ & $F=A\oplus B$ \\
\hline
0 & 1 & 1 & 1 & $F=A\overline{B}$ 减 1 & $F=A\overline{B}$ & $F=A\overline{B}$ \\
\hline
1 & 0 & 0 & 0 & $F=A$ 加 $AB$ & $F=A$ 加 $AB$ 加 1 & $F=A+B$ \\
\hline
1 & 0 & 0 & 1 & $F=A$ 加 $B$ & $F=A$ 加 $B$ 加 1 & $F=A\oplus B$ \\
\hline
1 & 0 & 1 & 0 & $F=(A+\overline{B})$ 加 $AB$ & $F=(A+\overline{B})$ 加 $AB$ 加 1 & $F=B$ \\
\hline
1 & 0 & 1 & 1 & $F=AB$ 减 1 & $F=AB$ & $F=AB$ \\
\hline
1 & 1 & 0 & 0 & $F=A$ 加 $A$ & $F=A$ 加 $A$ 加 1 & $F=1$ \\
\hline
1 & 1 & 0 & 1 & $F=(A+B)$ 加 $A$ & $F=(A+B)$ 加 $A$ 加 1 & $F=A+\overline{B}$ \\
\hline
1 & 1 & 1 & 0 & $F=(A+\overline{B})$ 加 $A$ & $F=(A+\overline{B})$ 加 $A$ 加 1 & $F=A+B$ \\
\hline
1 & 1 & 1 & 1 & $F=A$ 减 1 & $F=A$ & $F=A$ \\
\hline
\end{tabular}
\end{table}

\textbf{2. 三态门(74LS244)}

74LS244是一款八路三态缓冲器,用于总线隔离和数据传输控制:
\begin{itemize}
    \item 包含8个独立的缓冲器,分为两组,每组4个
    \item 每组有独立的使能控制端(1G、2G),低电平有效
    \item 输出端具有三态特性:高电平、低电平和高阻态
    \item 高阻态时输出端呈现高阻抗,可实现多个器件共享数据总线
    \item 主要用于隔离运算器与数据总线,防止总线冲突
\end{itemize}

其逻辑功能表如表~\ref{tab:74ls244}所示。

\begin{table}[htbp]
\centering
\caption{74LS244逻辑功能表}
\label{tab:74ls244}
\zihao{-5}
\begin{tabular}{|c|c|c|}
\hline
\multicolumn{2}{|c|}{\textbf{INPUTS}} & \textbf{OUTPUT} \\
\hline
\textbf{1G, 2G} & \textbf{D} & \\
\hline
L & L & L \\
\hline
L & H & H \\
\hline
H & X & H (Z) \\
\hline
\end{tabular}
\end{table}

\vspace{0.3cm}
注:L表示低电平,H表示高电平,X表示任意电平,Z表示高阻态。

\begin{figure}[htbp]
\centering
\begin{minipage}[t]{0.45\textwidth}
\centering
\includegraphics[width=0.9\textwidth]{74LS244}
\caption{74LS244引脚图}
\label{fig:74LS244}
\end{minipage}
\hfill
\begin{minipage}[t]{0.45\textwidth}
\centering
\includegraphics[width=0.9\textwidth]{74LS273}
\caption{74LS273引脚图}
\label{fig:74LS273}
\end{minipage}
\end{figure}

\textbf{3. 地址/数据锁存器(74LS273)}

74LS273是一款八路D触发器,用于数据锁存和同步:
\begin{itemize}
    \item 包含8个独立的D触发器,共用一个时钟信号(CLK)
    \item 在时钟上升沿锁存D端输入数据,并保持在Q端输出
    \item 具有异步清零端(CLR),低电平有效,可将所有输出清零
    \item 在本实验中用于锁存地址信号或数据信号,确保数据稳定传输
\end{itemize}

其逻辑功能表如表~\ref{tab:74ls273}所示(每个触发器)。

\begin{table}[htbp]
\centering
\caption{74LS273逻辑功能表(每个触发器)}
\label{tab:74ls273}
\zihao{-5}
\begin{tabular}{|c|c|c|c|}
\hline
\multicolumn{3}{|c|}{\textbf{INPUTS}} & \textbf{OUTPUT} \\
\hline
\textbf{CLEAR} & \textbf{CLOCK} & \textbf{D} & \textbf{Q} \\
\hline
L & X & X & L \\
\hline
H & $\uparrow$ & H & H \\
\hline
H & $\uparrow$ & L & L \\
\hline
H & L & X & Q$_0$ \\
\hline
\end{tabular}
\end{table}

\vspace{0.3cm}
注:L表示低电平,H表示高电平,X表示任意电平,$\uparrow$表示时钟上升沿,Q$_0$表示保持原状态。



\textbf{4. 反相器(74LS04)}

74LS04是一款六路反相器,用于信号取反:
\begin{itemize}
    \item 包含6个独立的非门,每个非门有一个输入端和一个输出端
    \item 输出信号与输入信号相反:输入高电平时输出低电平,反之亦然
    \item 在本实验中用于产生控制信号的反相信号,或进行逻辑电平转换
\end{itemize}

\begin{figure}[htbp]
\centering
\begin{minipage}[t]{0.45\textwidth}
\centering
\includegraphics[width=0.9\textwidth]{74LS04}
\caption{74LS04引脚图}
\label{fig:74LS04}
\end{minipage}
\hfill
\begin{minipage}[t]{0.45\textwidth}
\centering
\includegraphics[width=0.9\textwidth]{74HC4078}
\caption{74HC4078引脚图}
\label{fig:74HC4078}
\end{minipage}
\end{figure}

\textbf{5. 或/或非门(74HC4078)}

74HC4078是一款八输入或/或非门,用于多路信号的逻辑运算:
\begin{itemize}
    \item 具有8个输入端,可实现8路信号的或运算或或非运算
    \item 当任意输入端为高电平时,或门输出为高电平
    \item 或非门输出为或门输出的反相
    \item 在本实验中用于产生控制信号或进行多路信号判断
\end{itemize}

\subsection{实验电路}

本实验电路以74LS181算术逻辑运算器为核心,构建了一个完整的8位运算系统。
整个电路可分为运算器核心电路、数据输入输出电路、控制信号电路和标志位检测电路四个部分。
在运算器核心电路中,A、B两组输入数据通过地址/数据锁存器(74LS273)
锁存后送入74LS181的数据输入端,功能选择信号(S3-S0)和方式控制信号(M)由外部控制电路提供,
用于选择具体的运算类型。数据输入输出部分采用74LS273锁存器确保输入数据的稳定性,
运算结果则通过74LS244三态门输出到数据总线,其使能控制确保在需要时才将数据送上总线以避免总
线冲突。控制信号电路使用74LS04反相器产生所需的反相控制信号,74HC4078或/或非门用于组合多个
控制信号产生复杂的控制逻辑,包括锁存器时钟信号、三态门使能信号等。标志位检测电路负责产生运算
结果的状态标志,其中进位标志(CF)由74LS181的进位输出端(Cn+4)直接产生,零标志(ZF)通过
或非门检测运算结果是否全为0,符号标志(SF)则直接取运算结果的最高位(F3)。

实验电路的整体结构如图~\ref{fig:cri_sch}所示。

\begin{figure}[htbp]
\centering
\includegraphics[width=0.9\textwidth]{cri_sch}
\caption{实验电路原理图}
\label{fig:cri_sch}
\end{figure}

控制信号的不同组合对应不同的运算功能:

\begin{itemize}
    \item \textbf{算术运算模式(M=0)}:主要执行加法、减法、加1、减1等算术操作
    \item \textbf{逻辑运算模式(M=1)}:主要执行与、或、非、异或等逻辑操作
    \item \textbf{功能选择(S3-S0)}:具体定义运算的类型和特性
\end{itemize}

\section{实验内容与数据处理}

\subsection{实验步骤详述}

\begin{enumerate}
    \item DRA\_CLK=0, DRB\_CLK=0等待数据所存,$\overline{\text{SW\_BUS}}$=0 
    等待数据输入显示 $\overline{\text{ALU\_OE}}$ =1 
    关闭运算器输出
    \item 通过波拨码开关输入数据,此时两位十六进制数码管显示输入
    数据,输入后通过将DRA\_CLK置为1将当前2位十六进制数所存到DRA,重复上述步骤,输入数据后将
    DRB\_CLK置为1将当前2位十六进制数所存到DRB
    \item 通过设置ALU\_S0、ALU\_S1、ALU\_S2、ALU\_S3选择运算功能,通过设置ALU\_M选择算术运算或逻辑运算,
    通过设置ALU\_CN选择有进位或无进位运算
    \item 将$\overline{\text{SW\_BUS}}$=1 ,失能数据输入显示,将$\overline{\text{ALU\_OE}}$=0, 使能运算器输出
    观察并记录运算结果及标志位状态
\end{enumerate}

\subsection{实验一:无符号数运算验证}
令电路原理图中各个开关的初始状态为:DRA\_CLK=DRB\_CLK=0,$\overline{\text{SW\_BUS}}$=\\
$\overline{\text{SW\_BUS}}$
=1,(S3,S2,S1,S0,M,CN)=(1,1,1,1,1,1)。操作拨码开关,向数据暂存器DRA写入AAH,DRB写入55H
(即A=0xAAH,B=0x55H)。改变运算器的控制信号(S3,S2,S1,S0,M,CN)
的组合,运算器使能($\overline{\text{SW\_BUS}}$=0),观察运算器的输出和标志位,并填入下表
中,与理论值比较,验证74LS181的功能。

\begin{figure}[H]
\centering
\includegraphics[width=0.9\textwidth]{ex1.png}
\caption{实验一截图示例}
\label{fig:ex1}
\end{figure}

例:先将\texttt{AAH}锁存至\texttt{DRA},
再将\texttt{55H}锁存至\texttt{DRB}。
设置\texttt{S3S2S1S0}为0000,\texttt{ALU\_CN}设置为1无进位,
\texttt{M}设置为0,此时\texttt{ALU}执行加法运算,
输出结果为\texttt{AAH}。标志位\texttt{CF/ZF/SF}
=001。实验结果截图如图\ref{fig:ex1}所示。

实验结果记录如下图表~\ref{tab:ex1table}。

\begin{figure}[H]
\centering
\includegraphics[width=0.9\textwidth]{ex1table}
\caption{实验一结果记录表格示例}
\label{tab:ex1table}
\end{figure}


% 实验数据表格可在此处添加

\subsection{实验二:有符号数运算验证}
拨码开关向数据暂存器DRA、DRB分别打入有符号数+7AH,-75H
(即A=+0x7AH,B=-0x75H)。改变运算器的控制信号
(S3,S2,S1,S0,M,CN)的组合,运算器使能
($\overline{\text{ALU\_OE}}$=0),观察运算器的输出和标志
位,并填入上表中,与理论值比较,验证74LS181的功能。

\begin{figure}[H]
\centering
\includegraphics[width=0.8\textwidth]{ex2.png}
\caption{实验二截图示例}
\label{fig:ex2}
\end{figure}

例:先将7\texttt{AH}锁存至\texttt{DRA},
再将8\texttt{BH}锁存至\texttt{DRB}。设置
\texttt{S3S2S1S0}为0011,\texttt{CN}设置为0,
\texttt{M}设置为0,此时\texttt{ALU}执行加法运算,
输出结果为-5\texttt{H}(FBH)。标志位\texttt{CF/ZF/SF}
=110。实验结果截图如图\ref{fig:ex2}所示。

实验结果记录如下图表~\ref{tab:ex2table}。

\begin{figure}[H]
\centering
\includegraphics[width=0.8\textwidth]{ex2table}
\caption{实验二结果记录表格示例}
\label{tab:ex2table}
\end{figure}

\section{思考与讨论}

\subsection{思考题}
\begin{enumerate}
  \item 74LS181 组成的运算器通路,能否区分有符号数运算和无符号数运算?两者的运算过程有何不同?
  各自的数值表示范围是多少?

  74LS181 本身为组合逻辑 ALU,不区分数据的语义(有符号/无符号)——它只按二进制规则进行位运算与进位处理;两者的运算“过程”在硬件层面相同,但解释结果时不同。数值范围(以 8 位为例):无符号数 0..255($2^8-1$),有符号(补码)为 -128..+127。
  
  \item 在 74LS181 组成的运算器通路中,输入锁存器 
  DRA、DRB 的作用是什么?运算结果输出端连接的 74LS244 
  缓冲器的作用是什么?若去掉其中一个输入锁存器(使 74LS181
   输入直连总线),运算器还能正常工作吗?若去掉输出端 74LS24
   4(使 74LS181 输出直连总线),运算器还能正常工作吗?

  DRA/DRB 作为输入锁存器的作用是:数据同步与保持、时序隔离、避免在运算中总线变化导致读入不稳定数据。74LS244 作为输出缓冲器的作用是:提高驱动能力、实现三态输出以避免总线冲突、提供输出使能控制和信号整形。若去掉输入锁存器,74LS181 在理论上仍能工作(单片、总线上无其它驱动时),但在有总线共享或时序变化时会出现数据不稳定与竞态,实验系统将变得不可靠;若去掉输出缓冲器使输出直接连总线,若总线上只有该器件且不存在其他驱动,短时间内可能能观察到结果,但在多设备系统中会产生总线冲突、驱动能力不足或损坏器件,故不可行。

  \item 在 74LS181 进行无符号数运算时,标志位 SF 是否有意义
  ?在有符号数运算时,CF 的含义是否与无符号运算时一致?
  举例说明:做有符号数的减法(例如“ A 减 0”)时为何 CF 
  可能置位;做“A 加 0”时 CF 会置位吗?在什么情况下有符号数
  的加法会出现 CF 置位?

  在无符号运算中,SF(最高位)仅表示最高位的电平,对数值正负无意义,因此不能当作无符号溢出判据;有符号运算的溢出应由溢出标志(OF)判断,而不是单纯看 CF。CF 表示无符号运算的进位/借位(对有符号意义不同)。例如“ A 减 0”:减法通常由加上被减数与被减数的补码实现(A + (−0)),在进位链上可能产生进位/借位,从而使 CF 置位;“A 加 0”通常不会产生向更高位的进位,因此 CF 一般不置位。对于有符号数,加法发生溢出(OF 置位)的典型情况为:两个正数相加得到负数,或两个负数相加得到正数;此时 CF 可能为 1(若有进位穿出最高位),但判断有符号溢出应以 OF 为准,CF 与 OF 的意义并不完全一致。

\end{enumerate}


\subsection{有符号数与无符号数运算的本质区别}

通过对实验结果的深入分析,我们可以得出以下重要结论:

\textbf{运算过程的同一性:}
74LS181在硬件层面执行运算时,并不区分有符号数和无符号数。无论是哪种数据类型,运算器都按照相同的二进制规则进行处理。这种设计体现了计算机体系结构的一个重要原则:数据的解释权在于程序员或编译器,硬件只负责执行基本的二进制操作。

\textbf{结果解释的差异性:}
\begin{itemize}
    \item \textbf{无符号数}:结果直接解释为二进制数值,溢出通过进位标志CF判断
    \item \textbf{有符号数}:结果采用补码解释,溢出通过符号标志SF和溢出标志判断
\end{itemize}

\textbf{数值范围对比:}
\begin{itemize}
    \item 8位无符号数范围:0 到 255($2^8-1$)
    \item 8位有符号数范围:-128 到 +127(采用二进制补码表示)
\end{itemize}

这种范围差异直接影响了运算结果的正确性判断标准。

\subsection{标志位系统的深入理解}

\textbf{无符号数运算中的SF标志:}
在无符号数运算中,符号标志SF实际上没有实际意义。因为无符号数的最高位是数值位而非符号位,SF反映的只是最高位的状态,不能作为数值正负的判断依据。

\textbf{有符号数运算中的CF标志:}
\begin{itemize}
    \item CF标志在有符号数运算中的含义与无符号数不同
    \item 对于有符号数,CF不能直接作为溢出判断依据
    \item 需要结合溢出标志OF进行综合判断
\end{itemize}

\textbf{特殊运算情况分析:}
\begin{enumerate}
    \item \textbf{"A减0"操作}:在补码运算中,减法通过加补码实现。A减0等价于A加0的补码,由于0的补码是全0,但进位处理机制可能导致CF置位,这反映了补码运算的特殊性。
    
    \item \textbf{"A加0"操作}:通常不会使CF置位,因为不会产生向更高位的进位。
    
    \item \textbf{有符号数加法的CF置位条件}:当两个正数相加得到负数,或两个负数相加得到正数时,表明发生了溢出,此时CF可能置位,但更重要的是要检查OF标志。
\end{enumerate}

本次实验为我们后续学习计算机组成原理和体系结构奠定了重要的实践基础,对理解现代处理器中ALU的设计思想具有重要启发意义。
 
\end{spacing}
\end{document}